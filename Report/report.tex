\documentclass{article}

%\documentclass{report}
\usepackage{braket}
\usepackage{amsmath}
\usepackage{mathtools}
\usepackage{amssymb}
\usepackage{trsym}
\usepackage{pifont}
\usepackage{tcolorbox}
\usepackage[T1]{fontenc}
\usepackage[utf8]{inputenc}
\usepackage{ae,aecompl}
\usepackage[english]{babel}
\usepackage{amsfonts}
\usepackage[super]{nth}
\usepackage{float}
\usepackage{caption}
\usepackage{graphicx}
\usepackage{subcaption}
\usepackage{geometry}
\usepackage{csquotes}
\usepackage{tikz}
\usepackage{circuitikz}
\usepackage{listings}
\usepackage{bbm}
\usepackage{siunitx}
\usepackage{hyperref}
\makeatletter
\renewcommand\paragraph{\@startsection{paragraph}{4}{\z@}%
	{-2.5ex\@plus -1ex \@minus -.25ex}%
	{1.25ex \@plus .25ex}%
	{\normalfont\normalsize\bfseries}}
\makeatother
\setcounter{secnumdepth}{4} % how many sectioning levels to assign numbers to
\setcounter{tocdepth}{4}    % how many sectioning levels to show in ToC
\usetikzlibrary{decorations.pathmorphing}
\geometry{
	a4paper,
	total={150mm,237mm},
	left=30mm,
	top=25mm,
}

\graphicspath{{imgs/}}

\DeclareMathOperator{\var}{Var}

\title{Harmonic Oscillator}
\author{Benedikt Otto}


\begin{document}
	\maketitle
	\newpage
	\tableofcontents
	\newpage
	\begin{abstract}

	\end{abstract}
	\section{Introduction}
	\section{Theoretical Basis}
		To evaluate the behaviour of a quantum particle in a certain potential the path-integral method can be used.
		It corresponds to the generalisation of the principle of the least action in classical mechanics.
		%However the solution in quantum mechanics does not yield a defined path, rather 
		\begin{equation}
			K(a, b) = \int_a^b e^{iS/\hbar} \mathcal Dx(t)
			\label{eq:path_integral}
		\end{equation}
		In equation \ref{eq:path_integral} the integral is performed over the path from point $a$ to $b$.
		At first the positions of the particle at every time step is initialised to a constant value or a random distribution.
		Then for every time step a new position is drawn from a random distribution, in this case a gaussian distribution.
		If this change lowers the total action, calculated over all the time lattice points, the change is accepted.
		Otherwise, a linearly distributed random value is drawn in the range from 0 to 1 and compared to the exponent of the difference in the Action, divided by a $\hbar$.
		If the random variable is smaller than the exponential function, the value is accepted, otherwise is is rejected and the position at that time step not updated.
		This behaviour leads to the tendency to thermalise, but takes care of the quantum behaviour of the particle.
		\\
		The total action is given by equation \ref{eq:total_action}.
		\begin{equation}
			S = \epsilon \sum_{i=0}^{N - 1} \left(\frac{m(x_{i+1} - x_i)^2}{2\tau^2} + V(x_i)\right)
			\label{eq:total_action}
		\end{equation}
		Because the analysis is more convenient, often periodic boundary conditions are used.
		In this case one can define $x_0 = x_N$, so the right neighbour of $x_{N-1}$ is $x_0$.
		The left part of the equation corresponds to the kinetic energy.
		The velocity used is the average velocity, when the particle moves from site $x_i$ to $x_{i+1}$.
		Because every change only affects three terms, it is not necessary to recalculate the complete action for every change.
		\begin{equation}
			\begin{split}
				\Delta S(x_{i-1}, x_i, x'_i, x_{i+1}) =\\
				\tau\left(V(x_i) - V(x'_i) + m\frac{(x_{i-1} - x_i)^2 + (x_i - x_{i+1})^2 - (x_{i-1} - x'_i)^2 - (x'_i - x_{i+1})^2}{2\tau^2}\right)
			\end{split}
			\label{eq:delta_total_action}
		\end{equation}
		This rewrite reduces the complexity of the problem and improves performance.
		The time complexity per Metropolis iteration is thus reduced from $\mathcal O(n^2)$ to $\mathcal O(n)$, where n corresponds to the number of lattice cites.
	\section{Methods}
	\section{Results}
	\begin{figure}[htbp]
		\centering
		\includegraphics[width=0.4\textwidth]{../imgs/harmonic_oscillator_track/track_10010000_qq_1.pdf}
		\includegraphics[width=0.4\textwidth]{../imgs/harmonic_oscillator_track/track_10010000_qq_10.pdf}
		\\
		\includegraphics[width=0.4\textwidth]{../imgs/harmonic_oscillator_track/track_10010000_qq_20.pdf}
		\includegraphics[width=0.4\textwidth]{../imgs/harmonic_oscillator_track/track_10010000_qq_40.pdf}
		\\
		\includegraphics[width=0.4\textwidth]{../imgs/harmonic_oscillator_track/track_10010000_qq_80.pdf}
		\includegraphics[width=0.4\textwidth]{../imgs/harmonic_oscillator_track/track_10010000_qq_100.pdf}
		\caption{}
	\end{figure}
	\section{Discussion}
	\section{Summary and Outlook}
\end{document}