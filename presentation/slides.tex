\documentclass[aspectratio=169]{beamer}
%\documentclass[handout,aspectratio=169]{beamer}	%handout

\title[The Harmonic Oscillator]{The Harmonic Oscillator}
\author{Benedikt Otto}
\date{\printdate{31.03.2020}}
\institute{physics760: Computational Physics}

%\usetheme{Antibes}
%\usetheme{Berlin}
%\usetheme{Bergen}
%\usetheme{Boadilla}
\usetheme{Berkeley}	% gut
%\usetheme{Goettingen}
%\usetheme{Hannover}
%\usetheme{Marburg}
%\usetheme{Hannover}
%\usetheme{Hannover}
%\usecolortheme{beetle}
%\usecolortheme{dolphin}
%\usecolortheme{seahorse}
%\usecolortheme{sidebartab}
%\usecolortheme{whale}


\makeatletter
\beamer@headheight=1.5\baselineskip
\makeatother		% schmalere Titelzeile

\setbeamertemplate{footline} 
{ 
	\leavevmode% 
	\hbox{% 
		\begin{beamercolorbox}[wd=.25\paperwidth,ht=2.25ex,dp=1ex,center]{author in head/foot}% 
			\usebeamerfont{author in head/foot}Benedikt Otto%\insertshortauthor%~~(\insertshortinstitute) 
		\end{beamercolorbox}% 
		\begin{beamercolorbox}[wd=.5\paperwidth,ht=2.25ex,dp=1ex,center]{title in head/foot}% 
			\usebeamerfont{title in head/foot}\insertshorttitle
		\end{beamercolorbox}% 
		\begin{beamercolorbox}[wd=.15\paperwidth,ht=2.25ex,dp=1ex,center]{date in head/foot}% 
			\usebeamerfont{date in head/foot}\insertshortdate{}			
		\end{beamercolorbox}% 
		\begin{beamercolorbox}[wd=.1\paperwidth,ht=2.25ex,dp=1ex,right]{date in head/foot}% 
			\usebeamerfont{date in head/foot}
			\insertframenumber{}\hspace*{2ex} % / \inserttotalframenumber\hspace*{2ex} %\hspace*{2ex} 
		\end{beamercolorbox}}% 
	\vskip0pt% 
}

\newcommand{\appendixstyle}{
	\setbeamertemplate{footline} 
	{ 
		\leavevmode% 
		\hbox{% 
			\begin{beamercolorbox}[wd=.25\paperwidth,ht=2.25ex,dp=1ex,center]{author in head/foot}% 
				\usebeamerfont{author in head/foot}Benedikt Otto%\insertshortauthor%~~(\insertshortinstitute) 
			\end{beamercolorbox}% 
			\begin{beamercolorbox}[wd=.5\paperwidth,ht=2.25ex,dp=1ex,center]{title in head/foot}% 
				\usebeamerfont{title in head/foot}\insertshorttitle
			\end{beamercolorbox}% 
			\begin{beamercolorbox}[wd=.15\paperwidth,ht=2.25ex,dp=1ex,center]{date in head/foot}% 
				\usebeamerfont{date in head/foot}\insertshortdate{}			
			\end{beamercolorbox}% 
			\begin{beamercolorbox}[wd=.1\paperwidth,ht=2.25ex,dp=1ex,right]{date in head/foot}% 
				\usebeamerfont{date in head/foot}
				\insertframenumber{}*\hspace*{2ex} %\hspace*{2ex} 
		\end{beamercolorbox}}% 
		\vskip0pt% 
	}
}


%\lstnewenvironment{TeXlstlisting}{\lstset{language=[LaTeX]TeX}}{}
\usepackage{appendixnumberbeamer}
\usepackage{showexpl}
\usepackage{xcolor}
\usepackage{capt-of}
\usepackage{lipsum}
\usepackage{units}


\usepackage{listings}

%\usepackage[square,sort,comma,numbers]{natbib}
%\setbeamertemplate{bibliography item}{}
%\let\oldbibliography\thebibliography
%\renewcommand{\thebibliography}[1]{\vspace{-15px}\oldbibliography{#1}
%	\setlength{\itemsep}{-12pt}} %Reducing spacing in the bibliography.

%\setlength{\bibsep}{0pt}

%\usepackage{beamerthemesplit}

%\setcounter{tocdepth}{1}
%\setcounter{tocdepth}
\usepackage[british,UKenglish,USenglish,english]{babel}
\usepackage[utf8]{inputenc}
\usepackage{amsmath}
\usepackage{amsfonts}
\usepackage{amssymb}
\usepackage{epstopdf}
\usepackage[british]{isodate}
\usepackage{graphicx}
\usepackage{float}
\usepackage{caption}
\usepackage{graphicx}
\usepackage{subcaption}
\usepackage{geometry}
\usepackage{units}
\usepackage{amsmath}
\usepackage{csquotes}
\usepackage{tikz}
\usepackage{circuitikz}
\usepackage{listings}
\usepackage{paralist}
[decimalsymbol=comma]
\usepackage{siunitx}
\usepackage{color}
\definecolor{pink}{rgb}{1,0.5,0.5}
\makeatletter
%\renewcommand\paragraph{\@startsection{paragraph}{4}{\z@}%
%	{-2.5ex\@plus -1ex \@minus -.25ex}%
%	{1.25ex \@plus .25ex}%
%	{\normalfont\normalsize\bfseries}}
\makeatother
\setcounter{secnumdepth}{4} % how many sectioning levels to assign numbers to
\setcounter{tocdepth}{4}    % how many sectioning levels to show in ToC

\setbeamertemplate{itemize subitem}{$\bullet$}	% change subitem symbols

\beamertemplatenavigationsymbolsempty	%remove navigation symbols

\usepackage{hyperref}
\usetikzlibrary{decorations.pathmorphing}

%Define tikz bindings
\usepackage{tikz}
\usetikzlibrary{decorations.pathreplacing}
\usetikzlibrary{decorations.pathmorphing}
\usetikzlibrary{decorations.markings}
\usetikzlibrary{positioning, shapes, snakes, arrows}
\usepackage{scrextend}
\changefontsizes{13pt}		% groessere Schrift


%\tabrowsep10mm
\renewcommand{\arraystretch}{1.3}
\graphicspath{{Imgs/}}

% Titelseite
\begin{document}
\begin{frame}
	\titlepage
\end{frame}


\section{Motivation}
\begin{frame}
	\frametitle{Introduction}
	\begin{itemize}
		\item \textbf{Path integral method} method is the quantum mechanical generalisation of the \textbf{Principle of stationary Action}.
		\item Harmonic oscillator is well-understood
		\item Anharmonic oscillator serves as a toy model for the tunnelling effect
		\item Path-integral formalism used in more interesting systems as the QCD.
	\end{itemize}
\end{frame}

\section{Theory}
\begin{frame}
	\frametitle{Theory}
	\begin{itemize}
		\item Transition probability is $K(a, b) = \int_a^b e^{iS/\hbar} \mathcal Dx(t)$
		\item $\mathcal Dx(t)$ means integration over all paths starting at $a$ and resulting in $b$.
		\begin{itemize}
			\item Fast oscillations of the phase
			\item Infinite dimensional integral oven infinite boundaries
			\\ $\Rightarrow$ analytically generally not solvable
		\end{itemize}
		\item transition into \textbf{Euclidean} time $t \rightarrow it$
	\end{itemize}
\end{frame}

\begin{frame}
	\frametitle{Theory}
	\begin{itemize}
		\item $S = \tau \sum_i {V(x_i) + T(x_i, x_{i+1})}$
		\item only $\Delta S$ is important $\Rightarrow$ complete recalculation is not necessary
		\item $\Delta S = \tau \left(V(x_{i;new}) - V(x_{i;old}) + T(x_{i;new}, x_{i+1}) - T(x_{i;old}, x_{i+1})\right)$
	\end{itemize}
\end{frame}

\section{Methods}
\begin{frame}
	\frametitle{Methods}
	\begin{itemize}
		\item Metropolis-Hastings algorithm
		\item Initialisation of the (time) lattice with for example gaussian distributed random values
		\item Iterate repeatedly over all lattice sites
		\item Draw a new value for current lattice site
		\item Evaluate $\Delta S < 0$ $\Rightarrow$ accept change
		\item Else: Accept if $e^{-\Delta S / \hbar} > x$ for $x \in [0, 1]$ evenly
	\end{itemize}
\end{frame}


\begin{frame}
\frametitle{Results}

\end{frame}


\begin{frame}
\frametitle{}
\begin{center}
	Thank you for your attention!
\end{center}
\end{frame}

\section{References}
\begin{frame}
\setbeamertemplate{bibliography item}{\insertbiblabel}
\frametitle{References}
\newcounter{firstbib}
\vspace{-20px}
\begin{columns}[T]
	\begin{column}{0.49\textwidth}
		\begin{tiny}
			\begin{thebibliography}{99}
				\fontsize{6}{6}
				\bibitem{github} Public Github repository: Harmonic Oscillator, Benedikt Otto (s6beotto), \\\url{https://github.com/s6beotto/Harmonic-Oscillator}.
				\bibitem{latexrun} Public Github repository: latexrun, Austin Clements (aclements), \\\url{https://github.com/aclements/latexrun}.
				\bibitem{rushka_freericks} M. Rushka J. K. Freericks, \textit{A Completely Algebraic Solution of the Simple Harmonic Oscillator}, arXiv:1912.08355 [quant-ph] (2019).
				\bibitem{creutz_freedman} M. Creutz and B. Freedman, \textit{A statistical approach to quantum mechanics}, Annals of Physics, \textbf{132}, 427-462 (1981).
				\bibitem{rodgers_raes} R. Rodgers and L. Raes, \textit{Monte Carlo simulations of harmonic and anharmonic oscillators in discrete Euclidean time}, DESY Summer Student Programme (2014).
				\bibitem{bender} C. M. Bender and T. T. Wu, \textit{Anharmonic oscillator}, Phys. Rev. (2) \textbf{184} (1969), 1231–1260.
				\setcounter{firstbib}{\value{enumiv}}
			\end{thebibliography}
		\end{tiny}
	\end{column}
	\begin{column}{0.49\textwidth}
		\begin{tiny}
			\begin{thebibliography}{99}
				\fontsize{6}{6}
				\setcounter{enumiv}{\value{firstbib}}
			\end{thebibliography}
		\end{tiny}
	\end{column}
\end{columns}
\end{frame}


\appendixstyle

\appendix


\end{document}
