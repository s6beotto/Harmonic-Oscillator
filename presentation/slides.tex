\documentclass[aspectratio=169]{beamer}
%\documentclass[handout,aspectratio=169]{beamer}	%handout

\title[The Harmonic Oscillator]{The Harmonic Oscillator}
\author{Benedikt Otto}
\date{\printdate{31.03.2020}}
\institute{physics760: Computational Physics}

\include{template}
\include{packages_slides}
%Define tikz bindings
\usepackage{tikz}
\usetikzlibrary{decorations.pathreplacing}
\usetikzlibrary{decorations.pathmorphing}
\usetikzlibrary{decorations.markings}
\usetikzlibrary{positioning, shapes, snakes, arrows}
\usepackage{scrextend}
\changefontsizes{13pt}		% groessere Schrift


%\tabrowsep10mm
\renewcommand{\arraystretch}{1.3}
\graphicspath{{Imgs/}}

% Titelseite
\begin{document}
\begin{frame}
	\titlepage
\end{frame}


\section{Motivation}
\begin{frame}
	\frametitle{Introduction}
	\begin{itemize}
		\item \textbf{Path integral method} method is the quantum mechanical generalisation of the \textbf{Principle of stationary Action}.
		\item Harmonic oscillator is well-understood
		\item Anharmonic oscillator serves as a toy model for the tunnelling effect
		\item Path-integral formalism used in more interesting systems as the QCD.
	\end{itemize}
\end{frame}

\section{Theory}
\begin{frame}
	\frametitle{Theory}
	\begin{itemize}
		\item Transition probability is $K(a, b) = \int_a^b e^{iS/\hbar} \mathcal Dx(t)$
		\item $\mathcal Dx(t)$ means integration over all paths starting at $a$ and resulting in $b$.
		\begin{itemize}
			\item Fast oscillations of the phase
			\item Infinite dimensional integral oven infinite boundaries
			\\ $\Rightarrow$ analytically generally not solvable
		\end{itemize}
		\item transition into \textbf{Euclidean} time $t \rightarrow it$
	\end{itemize}
\end{frame}

\begin{frame}
	\frametitle{Theory}
	\begin{itemize}
		\item $S = \tau \sum_i {V(x_i) + T(x_i, x_{i+1})}$
		\item only $\Delta S$ is important $\Rightarrow$ complete recalculation is not necessary
		\item $\Delta S = \tau (V(x_{i;new}) - V(x_{i;old}) + $\\$ + T(x_{i-1}, x_{i;new}) + T(x_{i;new}, x_{i+1}) - T(x_{i-1}, x_{i;old}) - T(x_{i;old}, x_{i+1}))$
		\item Potential energy: $V(x) = \mu x^2 + \lambda x^4$
		\item Kinetic energy: $T(x_1, x_2) = \frac m2 \frac{(x_1 - x_2)^2}{\tau^2}$
	\end{itemize}
\end{frame}

\section{Methods}
\begin{frame}
	\frametitle{Methods}
	\begin{itemize}
		\item Metropolis-Hastings algorithm
		\item Initialisation of the (time) lattice with for example gaussian distributed random values
		\item Iterate repeatedly over all lattice sites
		\item Draw a new value for current lattice site
		\item Evaluate $\Delta S < 0$ $\Rightarrow$ accept change
		\item Else: Accept if $e^{-\Delta S / \hbar} > x$ for $x \in [0, 1]$ evenly
	\end{itemize}
\end{frame}


\section{Results}
\begin{frame}
	\frametitle{Verification: Harmonic oscillator}
	\vspace{-15px}
	\begin{columns}
		\begin{column}{0.49\textwidth}
			\begin{figure}[H]
				\centering
				\includegraphics[width=\textwidth]{../imgs/harmonic_oscillator_classical_limit/harmonic_oscillator_10_classical_limit.pdf}
				\caption{Classical limit harmonic oscillator.}
				\label{fig:harmonic_oscillator_classical_limit}
			\end{figure}
		\end{column}
		\begin{column}{0.49\textwidth}
			\begin{itemize}
				\item Condenses into classical minimum for $\hbar \rightarrow 0$
			\end{itemize}
		\end{column}
	\end{columns}
\end{frame}

\section{Results}
\begin{frame}
	\frametitle{Verification: Anharmonic oscillator}
	\vspace{-15px}
	\begin{columns}
		\begin{column}{0.49\textwidth}
			\begin{figure}[H]
				\centering\includegraphics[width=\textwidth]{../imgs/anharmonic_oscillator_classical_limit/anharmonic_oscillator_classical_limit.pdf}
			\caption{Classical limit anharmonic oscillator.}
			\label{fig:anharmonic_oscillator_classical_limit}
			\end{figure}
		\end{column}
		\begin{column}{0.49\textwidth}
			\begin{itemize}
				\item Initially prepared in the left minimum
				\item Condenses into classical minimum for $\hbar \rightarrow 0$
				\item For low $\hbar$ right minimum is not populated
			\end{itemize}
		\end{column}
	\end{columns}
\end{frame}

\begin{frame}
\frametitle{Measurements}

\end{frame}


\begin{frame}
\frametitle{}
\begin{center}
	Thank you for your attention!
\end{center}
\end{frame}

\section{References}
\begin{frame}
\setbeamertemplate{bibliography item}{\insertbiblabel}
\frametitle{References}
\newcounter{firstbib}
\vspace{-20px}
\begin{columns}[T]
	\begin{column}{0.49\textwidth}
		\begin{tiny}
			\begin{thebibliography}{99}
				\fontsize{6}{6}
				\bibitem{github} Public Github repository: Harmonic Oscillator, Benedikt Otto (s6beotto), \\\url{https://github.com/s6beotto/Harmonic-Oscillator}.
				\bibitem{latexrun} Public Github repository: latexrun, Austin Clements (aclements), \\\url{https://github.com/aclements/latexrun}.
				\bibitem{rushka_freericks} M. Rushka J. K. Freericks, \textit{A Completely Algebraic Solution of the Simple Harmonic Oscillator}, arXiv:1912.08355 [quant-ph] (2019).
				\bibitem{creutz_freedman} M. Creutz and B. Freedman, \textit{A statistical approach to quantum mechanics}, Annals of Physics, \textbf{132}, 427-462 (1981).
				\bibitem{rodgers_raes} R. Rodgers and L. Raes, \textit{Monte Carlo simulations of harmonic and anharmonic oscillators in discrete Euclidean time}, DESY Summer Student Programme (2014).
				\bibitem{bender} C. M. Bender and T. T. Wu, \textit{Anharmonic oscillator}, Phys. Rev. (2) \textbf{184} (1969), 1231–1260.
				\setcounter{firstbib}{\value{enumiv}}
			\end{thebibliography}
		\end{tiny}
	\end{column}
	\begin{column}{0.49\textwidth}
		\begin{tiny}
			\begin{thebibliography}{99}
				\fontsize{6}{6}
				\setcounter{enumiv}{\value{firstbib}}
			\end{thebibliography}
		\end{tiny}
	\end{column}
\end{columns}
\end{frame}


\appendixstyle

\appendix


\end{document}
